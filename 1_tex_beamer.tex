% lightly modified version of the legendary Bob Hall's template
% https://web.stanford.edu/~rehall/BeamerSkeleton.tex
% Document Preamble

%  Must be compiled with pdf Latex
\documentclass[12pt, aspectratio=169]{beamer}
%\documentclass[12pt,handout]{beamer}

\usepackage{/home/alal/Templates/boilerplate/lal_default_beamer}

% set universal bibliography (change to point to your master bibfile)
\bibliographystyle{plainnat}
\usepackage[backend=biber,
citestyle=authoryear
]{biblatex}
\addbibresource{/home/alal/Dropbox/MyLibrary.bib}

% If you want to print handouts
%\documentclass[handout]{beamer}

% Changing the fonts: this will make the slides more readable and the math look like regular tex math
\usefonttheme{serif}

% Titles will appear in Small Cap Serif
\usefonttheme{structuresmallcapsserif}

% -----------------------------------------
% Center the Frame Title
% -----------------------------------------

\setbeamertemplate{frametitle} {
\begin{centering}
\vspace{0.1in} \insertframetitle
\par
\end{centering}}

% -----------------------------------------
% Number the slides
% -----------------------------------------

\setbeamertemplate{footline}[frame number]

% -----------------------------------------
% Get rid of the irritating navigation bar
% -----------------------------------------

\setbeamertemplate{navigation symbols}{}

% -----------------------------------------
% Begin Document
% -----------------------------------------

\begin{document}

%  I separate each slide with a line:

% -----------------------------------------

%Information to be included in the title page:

\title{Title}
\subtitle{subtitle}

\author{Apoorva Lal}
\institute{Stanford}

\date{\today}
\titlegraphic{\includegraphics[width=0.17\textwidth,height=.3\textheight]{/home/alal/Templates/boilerplate/Stanford-Tree.jpg}}

%------------------------------------------------
\frame{\titlepage}


\begin{frame}{Overview}
\tableofcontents
\end{frame}

% -----------------------------------------

% /*
% ██ ███    ██ ████████ ██████   ██████
% ██ ████   ██    ██    ██   ██ ██    ██
% ██ ██ ██  ██    ██    ██████  ██    ██
% ██ ██  ██ ██    ██    ██   ██ ██    ██
% ██ ██   ████    ██    ██   ██  ██████
% */

\section{Introduction}
\begin{frame}
\frametitle{Evolution of venture's share}

$$
s_{i,i}=\frac{f_i}{f_i+v_i}
$$

$$
s_{i,j}=\frac{s_{i,j-1}v_j}{f_j+v_j}
$$

\end{frame}

% /*
% ████████ ██   ██ ███████  ██████  ██████  ██    ██
%    ██    ██   ██ ██      ██    ██ ██   ██  ██  ██
%    ██    ███████ █████   ██    ██ ██████    ████
%    ██    ██   ██ ██      ██    ██ ██   ██    ██
%    ██    ██   ██ ███████  ██████  ██   ██    ██
% */

\section{Theory}
\begin{frame}[t]\frametitle{Model}
\begin{center}
centered text
\end{center}

\begin{align*}
\maxi_{c_t,k_{t+1}} &  \sum_{t=1}^{\infty} \beta^t u(c_t)  \\
  s.t. & \enspace c_{t} + k_{t+1} \leq f(k_t) + (1-\delta)k_t
\end{align*}
Everyone does infinite horizon optimisation, right?
\end{frame}


% /*
% ███████ ███    ███ ██████  ██ ██████  ██  ██████ ███████
% ██      ████  ████ ██   ██ ██ ██   ██ ██ ██      ██
% █████   ██ ████ ██ ██████  ██ ██████  ██ ██      ███████
% ██      ██  ██  ██ ██      ██ ██   ██ ██ ██           ██
% ███████ ██      ██ ██      ██ ██   ██ ██  ██████ ███████
% */

\section{Empirics}
\begin{frame}[label=data_slide]\frametitle{Main Data}

\begin{columns}
\begin{column}{0.6\textwidth}
\begin{itemize}
   \item some text here some text here some text here some text here some text here
   \item \hyperlink{appendix_end}{\beamergotobutton{Details in appendix}}
\end{itemize}
\end{column}
\begin{column}{0.4\textwidth}
    \begin{center}
    Picture goes here
     % \includegraphics[width=1\textwidth]{picture.png}
     \end{center}
\end{column}
\end{columns}

\end{frame}

% -----------------------------------------
%  Example of a slide with a table

\begin{frame}
\frametitle{Data}

\centerline{
% \includegraphics[width=5in]{Table_1}
}

\end{frame}


% -----------------------------------------


\begin{frame}
\frametitle{Predictors}

\begin{itemize}
\item Number of this round
\item Amount raised in this round
\item Cumulative increase in the Wilshire index over the 2 years
preceding this round
\end{itemize}
\vspace{.5in}

Our specification has a complete set of
interactions by round number, except that we fit
the same coefficients for rounds 5 and higher.

\end{frame}

% -----------------------------------------
%  Example of a slide with a figure (handled just the same as a table)

\begin{frame}
\frametitle{GPs' earnings}

\centerline{
% \includegraphics<1>[width=4.in]{Fig_3}
}

\end{frame}

% /*
%  ██████  ██████  ███    ██  ██████ ██      ██    ██ ██████  ███████
% ██      ██    ██ ████   ██ ██      ██      ██    ██ ██   ██ ██
% ██      ██    ██ ██ ██  ██ ██      ██      ██    ██ ██   ██ █████
% ██      ██    ██ ██  ██ ██ ██      ██      ██    ██ ██   ██ ██
%  ██████  ██████  ██   ████  ██████ ███████  ██████  ██████  ███████
% */

\section{Conclusion}

\begin{frame}
\frametitle{Conclusion}
\end{frame}
% -----------------------------------------
%  Finish with dark screen again

\begin{frame}[t]\frametitle{Bibliography}
\printbibliography
\end{frame}

\begin{frame}
\frametitle{Thank You!}
\end{frame}

% /*
%  █████  ██████  ██████  ███████ ███    ██ ██████  ██ ██   ██
% ██   ██ ██   ██ ██   ██ ██      ████   ██ ██   ██ ██  ██ ██
% ███████ ██████  ██████  █████   ██ ██  ██ ██   ██ ██   ███
% ██   ██ ██      ██      ██      ██  ██ ██ ██   ██ ██  ██ ██
% ██   ██ ██      ██      ███████ ██   ████ ██████  ██ ██   ██
% */

\appendix
\begin{frame}[label=appendix_end]{Appendix land}
  \begin{itemize}
    \item[] Now you can make it clear you've done a shitload of work
      \begin{itemize}
      \item[]  without having to show everything! \hyperlink{data_slide}{\beamergotobutton{Back}}
      \end{itemize}
    \item[] You label a frame with the \texttt{[label=name]} option, and then point a link to it
    \item[] You can make an object a link using the \texttt{\textbackslash hyperlink\{label\}\{object\}} command
  \end{itemize}
\end{frame}
\end{document}
