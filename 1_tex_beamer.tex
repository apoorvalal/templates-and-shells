% lightly modified version of the legendary Bob Hall's template
% https://web.stanford.edu/~rehall/BeamerSkeleton.tex
% Document Preamble

%  Must be compiled with pdf Latex
\documentclass[12pt, aspectratio=169]{beamer}
%\documentclass[12pt,handout]{beamer}
% \usetheme{Rochester}

% \usepackage{/home/alal/Templates/boilerplate/lal_default_beamer}
% set universal bibliography (change to point to your master bibfile)
\bibliographystyle{plainnat}
\usepackage[backend=biber,
citestyle=authoryear
]{biblatex}
\addbibresource{/home/alal/Dropbox/MyLibrary2.bib}

% If you want to print handouts
%\documentclass[handout]{beamer}

% Changing the fonts: this will make the slides more readable and the math look like regular tex math
\usefonttheme{serif}

% Titles will appear in Small Cap Serif
\usefonttheme{structuresmallcapsserif}

% -----------------------------------------
% Center the Frame Title
% -----------------------------------------

\setbeamertemplate{frametitle} {
\begin{centering}
\vspace{0.1in} \insertframetitle
\par
\end{centering}
}

\setbeamertemplate{itemize items}[default]
\setbeamertemplate{enumerate items}[default]

% -----------------------------------------
% Number the slides
% -----------------------------------------

\setbeamertemplate{footline}[frame number]

% -----------------------------------------
% Get rid of the irritating navigation bar
% -----------------------------------------

\setbeamertemplate{navigation symbols}{}

% -----------------------------------------
% Begin Document
% -----------------------------------------

\begin{document}

%  I separate each slide with a line:

% -----------------------------------------

%Information to be included in the title page:

\title{Building State and Citizen:
How Tax Collection in Congo Engenders
Citizen Engagement with the State}
\subtitle{Jonathan Weigel}

\author{Discussant: Apoorva Lal}

\date{\today}

%------------------------------------------------
\frame{\titlepage}


\begin{frame}{Overview}
\tableofcontents
\end{frame}

% -----------------------------------------

% /*
% ██ ███    ██ ████████ ██████   ██████
% ██ ████   ██    ██    ██   ██ ██    ██
% ██ ██ ██  ██    ██    ██████  ██    ██
% ██ ██  ██ ██    ██    ██   ██ ██    ██
% ██ ██   ████    ██    ██   ██  ██████
% */

\section{Introduction}
\begin{frame}{Variation in State Capacity}

% \bi
% \item developing countries collect much less tax than developed countries(15\% of GDP versus 40\% of
% GDP
% \ei

\textbf{Historical Accounts of the development of State Capacity}

When European rulers in the early modern period began systematically
taxing their subjects, the people resisted, demanding public goods and
representation in return for tax compliance which in turn triggered
the co-evolution of tax compliance, citizen participation in politics,
and accountable governance.
\parencite{tillyCoercionCapitalEuropean1990,northConstitutionsCommitmentEvolution1989}

\vspace{5mm}

\textbf{Testable Proposition examined by the paper:}

\emph{tax collection increases citizen demand for political participation.}

Problem: Tax collection is not random; governments may strategically
target to maximise revenue and minimise distortions.

\end{frame}

% /*
% ███████ ███    ███ ██████  ██ ██████  ██  ██████ ███████
% ██      ████  ████ ██   ██ ██ ██   ██ ██ ██      ██
% █████   ██ ████ ██ ██████  ██ ██████  ██ ██      ███████
% ██      ██  ██  ██ ██      ██ ██   ██ ██ ██           ██
% ███████ ██      ██ ██      ██ ██   ██ ██  ██████ ███████
% */


\section{Empirics}
\begin{frame}{The Experiment}

\begin{itemize}
\item randomizing property tax collection across 431 neighborhoods
— covering roughly 33,000 properties — of Kananga, D.R. Congo (DRC),
in 2016.
\item selected 253 neighborhoods to receive the initial phase of
property tax collection campaign in the city. In treated
neighborhoods, tax collectors went door to door making in-person
appeals for the roughly \$2 property tax, which they collected on the
spot, issuing printed receipts to payers.
\item Control neighborhoods remained in the old declarative system:
citizens were supposed to pay at the bank themselves, but in practice
less than 1\% did.
\end{itemize}

\end{frame}

\begin{frame}{Empirical Setup}

\begin{itemize}
\item \emph{First Stage}: Did the campaign increase tax compliance :
$0.1 \%$ in controls to $10.3 \%$ in treatment = 100 fold increase
\item \emph{Reduced Form}: Did the experiment improve political
participation? Measures willingness to exert costly effort to
have a voice in the provincial government.
  \begin{itemize}
  \item  the government hosted a series of townhall meetings, in which
  officials and citizens had a dialog about taxation and public
  spending in Kananga.
  \item citizens could submit anonymous evaluations of the provincial
  government to a suggestion box whose contents would be shared with
  the governor and other top officials.
  \end{itemize}
\end{itemize}

\end{frame}

% /*
% ████████ ██   ██ ███████  ██████  ██████  ██    ██
%    ██    ██   ██ ██      ██    ██ ██   ██  ██  ██
%    ██    ███████ █████   ██    ██ ██████    ████
%    ██    ██   ██ ██      ██    ██ ██   ██    ██
%    ██    ██   ██ ███████  ██████  ██   ██    ██
% */

\section{Theory}
\begin{frame}[t]\frametitle{Model}
\begin{center}
centered text
\end{center}

\begin{align*}
max_{c_t,k_{t+1}} &  \sum_{t=1}^{\infty} \beta^t u(c_t)  \\
  s.t. & \enspace c_{t} + k_{t+1} \leq f(k_t) + (1-\delta)k_t
\end{align*}
Everyone does infinite horizon optimisation, right?

\end{frame}


% /*
%  ██████  ██████  ███    ██  ██████ ██      ██    ██ ██████  ███████
% ██      ██    ██ ████   ██ ██      ██      ██    ██ ██   ██ ██
% ██      ██    ██ ██ ██  ██ ██      ██      ██    ██ ██   ██ █████
% ██      ██    ██ ██  ██ ██ ██      ██      ██    ██ ██   ██ ██
%  ██████  ██████  ██   ████  ██████ ███████  ██████  ██████  ███████
% */

\section{Conclusion}

\begin{frame}
\frametitle{Conclusion}
\end{frame}
% -----------------------------------------
%  Finish with dark screen again


\begin{frame}
\frametitle{Thank You!}
\end{frame}

\begin{frame}[t]\frametitle{Bibliography}
\printbibliography
\end{frame}

% /*
%  █████  ██████  ██████  ███████ ███    ██ ██████  ██ ██   ██
% ██   ██ ██   ██ ██   ██ ██      ████   ██ ██   ██ ██  ██ ██
% ███████ ██████  ██████  █████   ██ ██  ██ ██   ██ ██   ███
% ██   ██ ██      ██      ██      ██  ██ ██ ██   ██ ██  ██ ██
% ██   ██ ██      ██      ███████ ██   ████ ██████  ██ ██   ██
% */

\appendix
\begin{frame}[label=appendix_end]{Appendix land}
  \begin{itemize}
    \item[] Now you can make it clear you've done a shitload of work
      \begin{itemize}
      \item[]  without having to show everything! \hyperlink{data_slide}{\beamergotobutton{Back}}
      \end{itemize}
    \item[] You label a frame with the \texttt{[label=name]} option, and then point a link to it
    \item[] You can make an object a link using the \texttt{\textbackslash hyperlink\{label\}\{object\}} command
  \end{itemize}
\end{frame}
\end{document}
