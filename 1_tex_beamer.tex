% \documentclass{beamer}
\documentclass[11pt, aspectratio=169]{beamer}
\usepackage[utf8]{inputenc}
\usepackage{bookmark}
\hypersetup{bookmarksdepth=4,bookmarksnumbered=true,bookmarksopen=true}
% either
\usetheme{metropolis}
% or {
% \usetheme[compress]{Berlin}
% \setbeamerfont{section in head/foot}{size=\tiny}

% hyperlinks and math shortcuts
\usepackage{/home/alal/Templates/boilerplate/lal_default_beamer}

% set universal bibliography (change to point to your master bibfile)
\bibliographystyle{plainnat}
\usepackage[backend=biber,
  citestyle=authoryear
]{biblatex}
\addbibresource{/home/alal/Dropbox/MyLibrary2.bib}

% fonts
\usepackage[T1]{fontenc}

\usepackage[no-math]{fontspec}
\setsansfont{Fira Sans Condensed}
\setmainfont{Fira Sans Condensed}

\beamertemplatenavigationsymbolsempty

%Information to be included in the title page:
\title{Title}
\subtitle{subtitle}
\author{Apoorva Lal}
\institute{Stanford}
\date{Date}

\begin{document}
%------------------------------------------------
\frame{\titlepage}
\begin{frame}
\frametitle{Overview}
\tableofcontents
% \tableofcontents[currentsection,subsubsectionstyle=hide] % optional for signposting
\end{frame}
%------------------------------------------------
%%%%%%%%%%%%%%%%%%%%%%%%%%%%%%%%%%%%%%%%%%%%%%%%%%%%%%%%%%%%%%%%%%%%%%

%------------------------------------------------
\subsection{Introduction}
\begin{frame}\frametitle{Introduction}
Stuff
\cite{angristMostlyHarmlessEconometrics2008}
\end{frame}
%------------------------------------------------
\subsection{Summary of Results}
\begin{frame}[t]\frametitle{Summary of Results}
Upshot
\end{frame}


%%%%%%%%%%%%%%%%%%%%%%%%%%%%%%%%%%%%%%%%%%%%%%%%%%%%%%%%%%%%%%%%%%%%%%
\section{Theory}
\tableofcontents[currentsection,subsubsectionstyle=hide]
%------------------------------------------------
\subsection{Model}
\begin{frame}[t]\frametitle{Model}
\begin{align*}
\maximise_{c_t,k_{t+1}} &  \sum_{t=1}^{\infty} \beta^t u(c_t)  \\
  s.t. & \enspace c_{t} + k_{t+1} \leq f(k_t) + (1-\delta)k_t
\end{align*}
Everyone does infinite horizon optimisation, right?
\end{frame}
%------------------------------------------------
% another slide here
%------------------------------------------------
%%%%%%%%%%%%%%%%%%%%%%%%%%%%%%%%%%%%%%%%%%%%%%%%%%%%%%%%%%%%%%%%%%%%%%
\section{Empirics}

%------------------------------------------------
\subsection{Data}
\begin{frame}[label=data_slide]\frametitle{Main Data}

\begin{columns}
\begin{column}{0.6\textwidth}
\begin{wideitemize}
   \item some text here some text here some text here some text here
   some text here
   \item \hyperlink{appendix_end}{\beamergotobutton{Details in appendix}}
\end{wideitemize}
\end{column}
\begin{column}{0.4\textwidth}
    \begin{center}
    Picture goes here
     % \includegraphics[width=1\textwidth]{picture.png}
     \end{center}
\end{column}
\end{columns}
\end{frame}
%------------------------------------------------
\subsection{Estimation}
\begin{frame}\frametitle{Estimation Framework}
\begin{align*}
\text{outcome}_{ict} = \alpha_i & + \sum_{k=0}^2 \beta_{t-k}^p
PPI_{ict-k} + \gamma_{ct} + \epsilon_{ict} \\
\text{outcome}_{ict} = \alpha_i & + \sum_{k=0}^2 \beta_{t-k}^p PPI_{ict-k} +
\sum_{k=0}^2 + \beta_{t-k}^m CPI_{ict-k} + \\ & \gamma_{c}\times
trend_t + \epsilon_{ict}
\end{align*}
\end{frame}
%------------------------------------------------
%%%%%%%%%%%%%%%%%%%%%%%%%%%%%%%%%%%%%%%%%%%%%%%%%%%%%%%%%%%%%%%%%%%%%%
\section{Conclusion}
\subsection{Conclusion}

\begin{frame}[t]\frametitle{Summary}
%
\begin{itemize}
\item Upshot
\item Closing remark
\end{itemize}

\end{frame}
%------------------------------------------------

\begin{frame}[t]\frametitle{Bibliography}
\printbibliography
\end{frame}
\begin{frame}[t]\frametitle{End}
\begin{center}
\Huge
Thank you!
\end{center}
\end{frame}
%%%%%%%%%%%%%%%%%%%%%%%%%%%%%%%%%%%%%%%%%%%%%%%%%%%%%%%%%%%%%%%%%%%%%%
\appendix
\begin{frame}[label=appendix_end]{Appendix land}
  \begin{wideitemize}
    \item[] Now you can make it clear you've done a shitload of work
      \begin{itemize}
      \item[]  without having to show everything! \hyperlink{data_slide}{\beamergotobutton{Back}}
      \end{itemize}
    \item[] You label a frame with the \texttt{[label=name]} option, and then point a link to it
    \item[] You can make an object a link using the \texttt{\textbackslash hyperlink\{label\}\{object\}} command
  \end{wideitemize}
\end{frame}
\end{document}
%
