\documentclass{article}
\usepackage{subfig,url}
\usepackage{"/home/alal/Templates/boilerplate/lal_default"}

% set universal bibliography (change to point to your master bibfile)
\bibliographystyle{plainnat}
\usepackage[backend=biber,
style=authoryear,
citestyle=authoryear,
]{biblatex}
\addbibresource{/home/alal/Dropbox/MyLibrary.bib}

%%% FILL THIS OUT
\newcommand{\studentname}{Apoorva Lal}
\newcommand{\coursename}{CourseName}
\newcommand{\suid}{apoorval}
\newcommand{\exerciseset}{Problem Set 1}
\newcommand{\univ}{Stanford}
\renewcommand{\theenumi}{\bf \Alph{enumi}}

% \theoremstyle{plain}
% \newtheorem{theorem}{Theorem}
% \newtheorem{lemma}[theorem]{Lemma}

\setlength{\parindent}{0pt}

\usepackage{fancyhdr}
\usepackage{booktabs}
\pagestyle{fancy}
\lhead{\univ; \coursename; \exerciseset}
\rhead[CO,LE]{\sffamily\bfseries\bfseries \studentname ; \suid}
\fancyfoot[LO,R]{\sffamily\bfseries\large \updateinfo}

\begin{document}
\begin{center}
\huge
\textsc{\coursename : \exerciseset}
\end{center}

\section*{Problem 1}
\begin{enumerate}
\item[3] %A
Your answer here.
\item[C\&B 5.6] %B
Next one here.
\item %C
42
\end{enumerate}

\section*{Problem 2}
\begin{enumerate}
\item %A
\item %B
\item %C
\item %D
\item %E
\end{enumerate}


\renewcommand{\mkbibnamefamily}[1]{\textsc{#1}} % IMPORTANT LINE TO BE COMMENTED
\printbibliography

\end{document}


