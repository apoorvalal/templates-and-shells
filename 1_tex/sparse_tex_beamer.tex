%  Must be compiled with pdf Latex
\documentclass[12pt, aspectratio=169]{beamer}
\usepackage{/home/alal/Templates/boilerplate/lal_default_beamer}

% set universal bibliography (change to point to your master bibfile)
% \bibliographystyle{plainnat}
% \usepackage[backend=biber,
% citestyle=authoryear
% ]{biblatex}
% \addbibresource{/home/alal/Dropbox/MyLibrary2.bib}

% Changing the fonts: this will make the slides more readable and the math look like regular tex math
\usecolortheme{beaver} % v stanford

% Titles will appear in Small Cap Serif
% \usefonttheme{structuresmallcapsserif}
\usepackage[no-math]{fontspec}
\setsansfont{IBM Plex Sans Condensed}
\setmainfont{IBM Plex Sans Condensed}

% -----------------------------------------
% Center the Frame Title
% -----------------------------------------

\setbeamertemplate{frametitle} {
\begin{centering}
\vspace{0.1in} \insertframetitle
\par
\end{centering}
}

\begin{document}

% -----------------------------------------
\title{Legislator Identity and Public Goods Provision: Evidence from
Indian Redistricting}
\author{Apoorva Lal}

\date{\today}

\frame{\titlepage}

% -----------------------------------------

% /*
% ██ ███    ██ ████████ ██████   ██████
% ██ ████   ██    ██    ██   ██ ██    ██
% ██ ██ ██  ██    ██    ██████  ██    ██
% ██ ██  ██ ██    ██    ██   ██ ██    ██
% ██ ██   ████    ██    ██   ██  ██████
% */

\begin{frame}{Introduction}

\begin{itemize}

\item Descriptive representation of the electorate in the legislature
  has been improved in many countries through a system of electoral
  policies such as quotas
\item Effectiveness of such policies in inducing better policy
  outcomes for less-represented groups is not well understood
\item Electoral representation especially important when constituency
  service includes basic public goods provision (Bussell (2019))
\end{itemize}

\textbf{Research Question:} Do different `types' of legislator perform
  different levels of constituency service, thereby resulting in
  economic outcomes?

\end{frame}

\begin{frame}[t]\frametitle{Redistricting varies legislator identity}

\bi
  \item Most democratic countries undergo a process of redistricting
  periodically
  \item Purported goal of equalising the number of voters in each
  constituencies in accordance with the normative goal of one person
  - one vote
  \item Different `types' of legislators perform varying degrees of
  constituency service (Anzia and Berry (2010), Bhalotra et al (2014))
  \item Redistricting combined with quota systems generates variation
  in `type' of legislator
  \item Permits \emph{within-unit} comparisons to evaluate effects of
  legislator identity.
  \item Existing literature has been unable to do this, and instead
  simply compares constituencies on a variety of outcomes after
  matching on a set of observable socio-economic characteristics
  [Jensenius (2015)]
\ei

\end{frame}

\begin{frame}[t]\frametitle{Reservations / Quotas in India}

\bi
  \item Seats in the Indian national and state legislatures are
  subject to a system of demographic reservations
  \item seats are categorised as `General', reserved for `Scheduled
    Castes', or reserved for `Scheduled tribes', i.e. only candidates
    from certain ethnic groups can run in specific electoral races,
    thereby fixing leader identity (on one axis) in that constituency
  \item Introduced by the Indian constitution following independence
  and was initially supposed to be in place for the first decade, with
  the goal of ensuring political participation by marginalised and
  under-represented groups.
\item has since been extended for seven decades by parliament, with
  the most recent extension in 2019 extending reservations to 2030
\ei


\end{frame}



% -----------------------------------------
\begin{frame}{Source of Variation - Flows}
\begin{figure}[tb]
  \centering
  \includegraphics[width=0.9\textwidth]{../../output/figures/village_ac_switches_alluvial_prez.pdf}
  \label{fig:switch_alluvial}
\end{figure}
\end{frame}

\begin{frame}
\begin{columns}
\begin{column}{0.5\textwidth}
\bi
  \item 2007-2008 re-drawing of boundaries was the first in nearly 3
  decades
  \item Delimitation Commission used specific census measures --
  Scheduled Caste and Scheduled Tribe population shares -- in their
  decision to redistrict villages.
  \item Two states - Jharkhand and Assam - were granted extensions for
  `technical' reasons
\ei
\end{column}
\begin{column}{0.5\textwidth}
\begin{figure}[tb]
  \centering
  \includegraphics[width=1.1\textwidth]{../../output/figures/village_ac_switches.png}
  \label{fig:switch_map}
\end{figure}

\end{column}
\end{columns}


\end{frame}


\begin{frame}{Data and Empirical Setup}

\bi
\item Data target: village-level panel dataset of public goods
  [currently night lights; soon-public works] with pre- and post-
  delimitation legislator type
  \scriptsize
\item electricity provision follows political cycle in many developing economies, including India (Min et al 2015, Min and Golden 2014)
\normalsize
  \be
  \item village level 2001 census data and shapefiles
  \item pre-and post-delimitation shapefiles for each state in India
  \item village level socio-economic outcomes from the Socioeconomic High-resolution Rural-Urban Geographic Dataset on India (which collates censuses)
  \item Spatial merge with harmonized DMSP luminosity series (1992 - 2013)
  \ee
\ei


\begin{align*}
y_{ijkt} & = \tau_1 \text{GEN-SC}_{ijkt} + \tau_2 \text{SC-GEN}_{ijkt} +
\tau_3 \text{GEN-ST}_{ijkt} + \tau_4 \text{ST-GEN}_{ijkt}  \\
& + \gamma_i(t) + \psi_{(jk)t} + \epsi_{ijkt}
\end{align*}

village $i$ in state $j$ with ST/SC decile $k$ at time $t$

\end{frame}

\begin{frame}
\tiny
\center
\input{../../output/tables/vil_nl_table_block.tex}
\end{frame}


\begin{frame}[t]\frametitle{Next Steps}
\be
  \item Attempt to reconcile specifications with and without time trends
  \item Cleaner event-studies [for particular treatment pairs]
  \item Public works programme provision [collection in progress]
  \item Direct measure of constituency service by MLAs
  \item Other village level outcomes
\ee
\end{frame}


\end{document}

