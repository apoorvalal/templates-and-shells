% math packages
\usepackage{
amscd,
amsfonts,
amsmath,
amssymb,
amsthm,
bm, % boldmath
dsfont,
cancel, % cancellation
latexsym,
mathtools,
}

\newcommand{\secend}{\vspace{3mm}}

% command shorthand
\newcommand{\eg}{e.g., \xspace}
\newcommand{\ie}{i.e.,\xspace}
\newcommand{\etc}{etc.\@\xspace}
\newcommand{\iid}{\emph{i.i.d.}\ }
\newcommand{\etal}{et.\ al.\ }
\newcommand{\D}{\displaystyle}
\newcommand{\ba}{\begin{array}}
\newcommand{\ea}{\end{array}}
\newcommand{\be}{\begin{enumerate}}
\newcommand{\ee}{\end{enumerate}}
\newcommand{\bi}{\begin{itemize}}
\newcommand{\ei}{\end{itemize}}
\newcommand{\bs}{\begin{align}\begin{split}\nonumber}
\newcommand{\bsnumber}{\begin{align}\begin{split}}
\newcommand{\es}{\end{split}\end{align}}
\newcommand{\fns}{\singlespace\footnotesize}

% math shorthand
% vertical equal prefix
\newcommand{\verteq}{\rotatebox{90}{$\,=$}}
% vertical equal to
\newcommand{\equalto}[2]{\underset{\scriptstyle\overset{\mkern4mu\verteq}{#2}}{#1}}
% nullspace
\newcommand{\nulls}{\mathrm{null}}
% range
\newcommand{\range}{\mathrm{range}}
% maximise
\newcommand{\maximise}{\operatornamewithlimits{max}}
% minimise
\newcommand{\minimise}{\operatornamewithlimits{min}}
% maximiser
\newcommand{\argmax}{\operatornamewithlimits{arg\,max}}

% such that (inside math mode)
\newcommand{\suchthat}{\text{ s.t. }}

% indicator function
\newcommand*\Indic[1]{\mathds{1}_{#1}}

% big parentheses
\newcommand*\Bigpar[1]{\left( #1 \right )}

% convergence in probability sideways
\def\inprobLOW{\rightarrow_p}
% convergence in probability
\def\inprobHIGH{\,{\buildrel p \over \rightarrow}\,}
% convergence in probability 2
\def\inprob{\,{\inprobHIGH}\,}
% convergence in distribution
\def\indist{\,{\buildrel d \over \rightarrow}\,}

% equality in distribution
\def\eqdist{\,{\buildrel d \over = }\,}

% independence (bench)
\newcommand\indep{\protect\mathpalette{\protect\independenT}{\perp}}
\def\independenT#1#2{\mathrel{\rlap{$#1#2$}\mkern5mu{#1#2}}}

% ellipsis
\newcommand{\tto}{,\ldots,}

% blackboard F
\def\Function{\mathbb{F}}

% Lagrangian
\def\Lagr{\mathcal{L}}

% n-dimensional Real
\def\Realn{\mathbb{R}^n}

% k-dimensional Real
\def\Rk{\mathbb{R}^k}

% real with argument
\newcommand{\Realm}[1]{\mathbb{R}^{#1}}

% P_n
\def\Probn{\mathbb{P}_n}

\def\rel{\,{\buildrel R \over \sim}\,}
% generic m x n matrix
\newcommand{\gmatrix}[1]{\begin{pmatrix} {#1}_{11} & \cdots &{#1}_{1n} \\ \vdots & \ddots & \vdots \\ {#1}_{m1} & \cdots &{#1}_{mn} \end{pmatrix}}

% Likelihood
\newcommand{\Likl}{\mathcal{L}}

% inner product
\newcommand{\iprod}[2]{\left\langle {#1} , {#2} \right\rangle}
% vector norm
\newcommand{\norm}[1]{\left\Vert {#1} \right\Vert}

% absolute value
\newcommand{\abs}[1]{\left\vert {#1} \right\vert}
% linalg misc
\renewcommand{\det}{\mathrm{det}}
\newcommand{\rank}{\mathrm{rank}}
\newcommand{\spn}{\mathrm{span}}
\newcommand{\row}{\mathrm{Row}}
\newcommand{\col}{\mathrm{Col}}
\renewcommand{\dim}{\mathrm{dim}}
% weakly prefer
\newcommand{\prefeq}{\succeq}
% strictly prefer
\newcommand{\pref}{\succ}
% sequence
\newcommand{\seq}[1]{\{{#1}_n \}_{n=1}^\infty }
% single arrow
\renewcommand{\to}{{\rightarrow}}
% double arrow
\newcommand{\corres}{\overrightarrow{\rightarrow}}
% evaluate at
\newcommand*\Eval[2]{\left.#1\right\rvert_{#2}}
% evaluate definite integral
\newcommand*\IntEval[3]{\left.#1\right\rvert_{#2}^{#3}}
% expectation
\newcommand{\Exp}[1]{\mathbb{E}\left[#1\right]}
% expectation at time
\newcommand{\Expt}[2]{\mathbb{E}_{#1}\left[#2\right]}
% variance
\newcommand{\Var}[1]{\mathbb{V}\left[#1\right]}
% Probability
\newcommand{\Prob}[1]{\mathbf{Pr}\left(#1\right)}
\newcommand{\F}{\mathscr{F}}
% \newcommand{\f}{\widehat{\eta}}
%Blackboard Letters
\newcommand{\R}{\ensuremath{\mathbb{R}}}
\newcommand{\Z}{\ensuremath{\mathbb{Z}}}
\newcommand{\Q}{\mathbb{Q}}
\newcommand{\N}{\mathbb{N}}
\newcommand{\W}{\mathbb{W}}
\newcommand{\Qoft}{\mathbb{Q}(t)}  %use in linux

\newcommand\frakfamily{\usefont{U}{yfrak}{m}{n}}
\DeclareTextFontCommand{\textfrak}{\frakfamily}

% Fractions
\newcommand{\fof}{\frac{1}{4}}  %oneforth
\newcommand{\foh}{\frac{1}{2}}  %onehalf
\newcommand{\fot}{\frac{1}{3}}  %onethird
\newcommand{\fop}{\frac{1}{\pi}}    %1/pi
\newcommand{\ftp}{\frac{2}{\pi}}    %2/pi
\newcommand{\fotp}{\frac{1}{2 \pi}} %1/2pi
\newcommand{\fotpi}{\frac{1}{2 \pi i}}
\newcommand{\cm}{c_{\text{{\rm mean}}}}
\newcommand{\cv}{c_{\text{{\rm variance}}}}


% math shorthand
% vertical equal prefix

% minimiser
\newcommand{\argmin}{\operatornamewithlimits{arg\,min}}
% convergence in probability sideways
\def\inprobLOW{\rightarrow_p}
% convergence in probability
\def\inprobHIGH{\,{\buildrel p \over \rightarrow}\,}
% convergence in probability 2
\def\inprob{\,{\inprobHIGH}\,}
% convergence in distribution
\def\indist{\,{\buildrel d \over \rightarrow}\,}

% such that
\def\ST{\text{ s.t. }}

% if
\def\IF{\text{ if }}

% definition bench
\def\deq{\coloneqq}


% blackboard F
\def\Function{\mathbb{F}}

% Natural
\def\Nat{\mathbb{N}}
% Integers
\def\Int{\mathbb{Z}}
% Reals
\def\Real{\mathbb{R}}
% Rationals
\def\Rat{\mathbb{Q}}
% Complex
\def\Cplx{\mathbb{C}}

% n-dimensional Real
\def\Realn{\mathbb{R}^n}
% expectation_n
\def\Expn{\mathbb{E}_n}
% P_n
\def\Probn{\mathbb{P}_n}

\def\rel{\,{\buildrel R \over \sim}\,}



\renewcommand{\det}{\mathrm{det}}
\renewcommand{\dim}{\mathrm{dim}}
% single arrow
\renewcommand{\to}{{\rightarrow}}

\newcommand{\mc}[1]{\mathcal{#1}}
\def\mbf#1{\mathbf{#1}}
\def\mrm#1{\mathrm{#1}}
\def\mbi#1{\boldsymbol{#1}} % Bold and italic (math bold italic)
\def\v#1{\mbi{#1}} % Vector notation


\newcommand{\lone}[1]{\norm{#1}_1} % l1 norm
\newcommand{\ltwo}[1]{\norm{#1}_2} % l2 norm
\newcommand{\pnorm}[1]{\norm{#1}_p} % p-norm
\newcommand{\linf}[1]{\norm{#1}_\infty} % l-infinity norm
\newcommand{\dnorm}[1]{\norm{#1}_*} % Dual norm
\newcommand{\lfro}[1]{\left\|{#1}\right\|_{\rm Fr}} % Frobenius norm
\newcommand{\matrixnorm}[1]{\left|\!\left|\!\left|{#1}
  \right|\!\right|\!\right|} % Matrix norm with three bars
\newcommand{\matrixnorms}[1]{|\!|\!|{#1}|\!|\!|} % Small matrix norm
\newcommand{\opnorm}[1]{\matrixnorm{#1}_{\rm op}}
\newcommand{\opnorms}[1]{\matrixnorms{#1}_{\rm op}}
\newcommand{\normbigg}[1]{\bigg\|{#1}\bigg\|} % A norm with 1 argument and bigg
                                              % brackets.
\newcommand{\lonebigg}[1]{\normbigg{#1}_1} % l1 norm
\newcommand{\ltwobigg}[1]{\normbigg{#1}_2} % l2 norm
\newcommand{\linfbigg}[1]{\normbigg{#1}_\infty} % l-infinity norm
\newcommand{\norms}[1]{\|{#1}\|} % A norm with 1 argument and normal (small)
                                 % brackets.
\newcommand{\lones}[1]{\norms{#1}_1} % l1 norm with small brackets
\newcommand{\ltwos}[1]{\norms{#1}_2} % l2 norm with small brackets
\newcommand{\linfs}[1]{\norms{#1}_\infty} % l-infinity norm with small brackets

\newcommand{\hinge}[1]{\left[{#1}\right]_+}

\newcommand{\defeq}{\vcentcolon=}
\newcommand{\eqdef}{=\vcentcolon}

\newcommand{\what}[1]{\widehat{#1}} % Wide hat
\newcommand{\wt}[1]{\widetilde{#1}} % Wide tilde
\newcommand{\wb}[1]{\overline{#1}} % Wide bar

\newcommand{\half}{\frac{1}{2}}

\newcommand{\<}{\left\langle} % Angle brackets
\renewcommand{\>}{\right\rangle} % End angle brackets

\renewcommand{\iff}{\Leftrightarrow}
\renewcommand{\choose}[2]{\binom{#1}{#2}}  % Choose
\newcommand{\chooses}[2]{{}_{#1}C_{#2}}  % Small choose

%%%% Probability symbols and associated distances %%%%

\newcommand{\E}{\mathbb{E}} % Expectation symbol
\renewcommand{\P}{\mathbb{P}} % Probability symbol
\newcommand{\var}{{\rm Var}} % Variance
\newcommand{\cov}{\mathop{\rm Cov}} % Covariance
\newcommand{\simiid}{\stackrel{\rm iid}{\sim}}
\newcommand{\openleft}[2]{\left({#1},{#2}\right]} % Interval open on left
\newcommand{\openright}[2]{\left[{#1},{#2}\right)} % Interval open on right

\newcommand{\indic}[1]{\mbf{1}\left\{#1\right\}} % Indicator function

% Distances between probability measures
\newcommand{\tvnorm}[1]{\norm{#1}_{\rm TV}} % Total variation
\newcommand{\tvnorms}[1]{\norms{#1}_{\rm TV}}
\newcommand{\dkl}[2]{D_{\rm kl}\left({#1} |\!| {#2}\right)} % KL divergence
\newcommand{\dkls}[2]{D_{\rm kl}({#1} |\!| {#2})}  % Small KL-divergence
\newcommand{\dchi}[2]{D_{\chi^2}\left({#1} |\!| {#2}\right)}  % chi^2-divergence
\newcommand{\fdiv}[2]{D_f\left({#1} |\!| {#2}\right)} % f divergence
\newcommand{\kl}[2]{D_{\rm kl}\left({#1} |\!| {#2} \right)}
\newcommand{\dhel}{d_{\rm hel}}  % Hellinger distance
\newcommand{\helaff}{A_{\rm hel}}  % Hellinger affinity

% Convergence of random variables
\newcommand{\cd}{\stackrel{d}{\rightarrow}}
\newcommand{\cas}{\stackrel{a.s.}{\rightarrow}}
\newcommand{\cp}{\stackrel{p}{\rightarrow}}

% Probability distributions
% \newcommand{\normal}{\mathsf{N}}  % Normal distribution
\newcommand*\normal[1]{\mathsf{N} \left( #1 \right )}
% \newcommand{\unif}{\mathsf{U}}  % Uniform distribution
\newcommand*\unif[1]{\mathsf{U} \left( #1 \right )}
% beta
\newcommand*\Bdist[1]{\mathsf{Beta} \left( #1 \right )}
% dirichlet
\newcommand*\Diri[1]{\mathsf{Dir} \left( #1 \right )}
% gamma
\newcommand*\Gdist[1]{\mathsf{Gamma} \left( #1 \right )}
% inv chi squared
\newcommand*\InvChi[1]{\mathsf{Inv}-\chi^2 \left( #1 \right )}

% Simple floor/ceiling stuff
\newcommand{\floor}[1]{\left\lfloor{#1} \right\rfloor}
\newcommand{\ceil}[1]{\left\lceil{#1} \right\rceil}

\providecommand{\argmax}{\mathop{\rm argmax}} % Defining math symbols
\providecommand{\argmin}{\mathop{\rm argmin}}
\providecommand{\soup}{\mathop{\rm sup}}

%  linalg stuff
\providecommand{\dom}{\mathop{\rm dom}}
\providecommand{\diag}{\mathop{\rm diag}}
\providecommand{\tr}{\mathop{\rm tr}}
\providecommand{\abs}{\mathop{\rm abs}}
\providecommand{\card}{\mathop{\rm card}}
\providecommand{\sign}{\mathop{\rm sign}}
\providecommand{\cl}{\mathop{\rm cl}}
\providecommand{\interior}{\mathop{\rm int}}
\providecommand{\conv}{\mathop{\rm Conv}}
\providecommand{\relint}{\mathop{\rm relint}}
\providecommand{\vol}{\mathop{\rm Vol}}
\providecommand{\supp}{\mathop{\rm supp}}

\providecommand{\minimize}{\mathop{\rm minimize}}
\providecommand{\maximize}{\mathop{\rm maximize}}
\providecommand{\subjectto}{\mathop{\rm subject\;to}}

% Proof environments
% The Theorems are numbered consecutively
% Lemmas are numbered by section, and observations, claims, facts, and
% assumptions take their numbering. Propositions and definitions have their
% own numbering by section.

% \theoremstyle{plain}
\theoremstyle{definition}
\newtheorem{thm}{Theorem}[section]
\newtheorem{assumption}{Assumption}
\newtheorem{cla}[thm]{Claim}
\newtheorem{conjecture}[thm]{Conjecture}
\newtheorem{conj}[thm]{Conjecture}
\newtheorem{cor}[thm]{Corollary}
\newtheorem{defi}[thm]{Definition}
\newtheorem{egg}[thm]{Example}
\newtheorem{exe}[thm]{Exercise}
\newtheorem{hypothesis}[thm]{Hypothesis}
\newtheorem{lem}[thm]{Lemma}
\newtheorem{prob}[thm]{Problem}
\newtheorem{proj}[thm]{Research Project}
\newtheorem{proposition}[thm]{Proposition}
\newtheorem{prop}[thm]{Proposition}
\newtheorem{que}[thm]{Question}
\newtheorem{rek}[thm]{Remark}
\newtheorem{rem}[thm]{Remark}
\newtheorem{X}{X}[section]


\renewenvironment{proof}{\noindent{\bf Proof}\hspace*{1em}}{\qed\bigskip\\}
\newenvironment{proof-sketch}{\noindent{\bf Sketch of Proof}
  \hspace*{1em}}{\qed\bigskip\\}
\newenvironment{proof-idea}{\noindent{\bf Proof Idea}
  \hspace*{1em}}{\qed\bigskip\\}
\newenvironment{proof-of-lemma}[1][{}]{\noindent{\bf Proof of Lemma {#1}}
  \hspace*{1em}}{\qed\bigskip\\}
\newenvironment{proof-of-proposition}[1][{}]{\noindent{\bf
    Proof of Proposition {#1}}
  \hspace*{1em}}{\qed\bigskip\\}
\newenvironment{proof-of-theorem}[1][{}]{\noindent{\bf Proof of Theorem {#1}}
  \hspace*{1em}}{\qed\bigskip\\}
\newenvironment{inner-proof}{\noindent{\bf Proof}\hspace{1em}}{
  $\bigtriangledown$\medskip\\}
%% \newenvironment{proof-of-lemma}[1][{}]{\noindent{\bf Proof of Lemma {#1}}
%%   \hspace*{1em}\renewcommand{\qedsymbol}{$\bigtriangledown$}}{\qed\bigskip\\}
\newenvironment{proof-attempt}{\noindent{\bf Proof Attempt}
  \hspace*{1em}}{\qed\bigskip\\}
\newenvironment{proofof}[1]{\noindent{\bf Proof} of {#1}:
  \hspace*{1em}}{\qed\bigskip\\}
\newenvironment{remark}{\noindent{\bf Remark}
  \hspace*{1em}}{\bigskip}

