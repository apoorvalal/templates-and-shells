\documentclass[12pt,reqno]{amsart} % add titlepage param for separate title page
\usepackage[margin=1in]{geometry}
\usepackage[foot]{amsaddr}
\usepackage[utf8]{inputenc}
\usepackage[english]{babel}
\usepackage{amsmath,amsthm,amssymb,comment}
\usepackage{braket}
\usepackage{mathtools}
\usepackage{caption}
\usepackage{mathrsfs}
\usepackage{latexsym}
\usepackage[dvips]{graphics}
\usepackage{epsfig}
\usepackage{amsmath,amsfonts,amsthm,amssymb,amscd}
\usepackage{color}
\usepackage{hyperref}
\usepackage{url}
%\usepackage{breakurl}
\newcommand{\bburl}[1]{\textcolor{blue}{\url{#1}}}

\hypersetup{breaklinks=true,
            bookmarks=true,
            pdfauthor={Apoorva Lal (Stanford)},
             pdfkeywords = {},
            colorlinks=true,
            citecolor=blue,
            urlcolor=blue,
            linkcolor=blue,
            pdfborder={0 0 0}}
\urlstyle{same}  % don't use monospace font for urls


\usepackage{tikz}
\usepackage{tkz-tab}
\usepackage{tkz-graph}
\usetikzlibrary{shapes.geometric,positioning}
%\newcommand{\burl}[1]{\textcolor{blue}{\url{#1}}}
\newcommand{\blue}[1]{\textcolor{blue}{(\bf{#1})}}
\newcommand{\emaillink}[1]{\textcolor{blue}{\href{mailto:#1}{#1}}}
\newcommand{\burl}[1]{\textcolor{blue}{\url{#1}}}
\newcommand{\fix}[1]{\textcolor{red}{\textbf{ (#1)\normalsize}}}
\newcommand{\fixed}[1]{\textcolor{green}{~\\ \textbf{#1\normalsize}}\\}
\newcommand{\ind}{\otimes}
\newcommand{\witi}{\widetilde}
\newcommand{\ch}{{\bf 1}}
\newcommand{\dt}[1]{\witi{\witi #1}}
\newcommand{\ol}[1]{\overline{#1}}
\newcommand{\lr}[1]{\left\lfloor#1\right\rfloor}
\newcommand{\eqd}{\overset{\footnotesize{d}}{=}}
\newcommand{\calf}{{\mathcal F}}
\newcommand{\cal}{\mathcal}

\renewcommand{\theequation}{\thesection.\arabic{equation}}
\numberwithin{equation}{section}

\newtheorem{thm}{Theorem}[section]
\newtheorem{conj}[thm]{Conjecture}
\newtheorem{cor}[thm]{Corollary}
\newtheorem{lem}[thm]{Lemma}
\newtheorem{prop}[thm]{Proposition}
\newtheorem{exa}[thm]{Example}
\newtheorem{defi}[thm]{Definition}
\newtheorem{exe}[thm]{Exercise}
\newtheorem{que}[thm]{Question}
\newtheorem{prob}[thm]{Problem}
\newtheorem{cla}[thm]{Claim}
\newtheorem{proj}[thm]{Research Project}

\theoremstyle{plain}
\newtheorem{X}{X}[section]
\newtheorem{corollary}[thm]{Corollary}
\newtheorem{definition}[thm]{Definition}
\newtheorem{example}[thm]{Example}
\newtheorem{lemma}[thm]{Lemma}
\newtheorem{proposition}[thm]{Proposition}
\newtheorem{theorem}[thm]{Theorem}
\newtheorem{conjecture}[thm]{Conjecture}
\newtheorem{hypothesis}[thm]{Hypothesis}
\newtheorem{rem}[thm]{Remark}
\newtheorem{rek}[thm]{Remark}

\newtheorem{remark}[thm]{Remark}

\renewcommand\thesection{\arabic{section}}
\newcommand{\F}{\mathscr{F}}
\newcommand{\f}{\widehat{\eta}}


%%%%%%%%%%%%%% Dirichlet characters
\newcommand{\Norm}[1]{\frac{#1}{\sqrt{N}}}

\newcommand{\sumii}[1]{\sum_{#1 = -\infty}^\infty}
\newcommand{\sumzi}[1]{\sum_{#1 = 0}^\infty}
\newcommand{\sumoi}[1]{\sum_{#1 = 1}^\infty}

\newcommand{\eprod}[1]{\prod_p \left(#1\right)^{-1}}
%$(s,b)$-Generacci
\newcommand{\sbg}{(s,b)\text{-Generacci}}
\newcommand{\sbs}{(s,b)\text{-Generacci\ sequence}}
\newcommand{\fqs}{\text{Fibonacci\ Quilt\ sequence}}
\newcommand{\fq}{\text{Fibonacci\ Quilt}}

\newcommand\st{\text{s.t.\ }}
\newcommand\be{\begin{equation}}
\newcommand\ee{\end{equation}}
\newcommand\bee{\begin{equation*}}
\newcommand\eee{\end{equation*}}
\newcommand\bea{\begin{eqnarray}}
\newcommand\eea{\end{eqnarray}}
\newcommand\beae{\begin{eqnarray*}}
\newcommand\eeae{\end{eqnarray*}}
\newcommand\bi{\begin{itemize}}
\newcommand\ei{\end{itemize}}
\newcommand\ben{\begin{enumerate}}
\newcommand\een{\end{enumerate}}
\newcommand\bc{\begin{center}}
\newcommand\ec{\end{center}}
\newcommand\ba{\begin{array}}
\newcommand\ea{\end{array}}

\newcommand{\mo}{\text{mod}\ }

\newcommand\ie{{i.e.,\ }}
\newcommand{\tbf}[1]{\textbf{#1}}
\newcommand\CF{{Continued Fraction}}
\newcommand\cf{{continued fraction}}
\newcommand\cfs{{continued fractions}}
\newcommand\usb[2]{\underset{#1}{\underbrace{#2}}}


%kmin and kmax
\newcommand{\kmin}[1]{k_{\min}(#1)}
\newcommand{\kmax}[1]{k_{\max}(#1)}
\newcommand{\kminm}{K_{\min}(m)}
\newcommand{\kmaxm}{K_{\max}(m)}
\newcommand{\B}{\mathcal{B}}

% General Symbols
\def\notdiv{\ \mathbin{\mkern-8mu|\!\!\!\smallsetminus}}
\newcommand{\done}{\Box} %use in linux
%\newcommand{\umess}[2]{\underset{(#1)}{\underbrace{#2}}}
\newcommand{\umessclean}[2]{\underset{=#1}{\underbrace{#2}}}

%Blackboard Letters
\newcommand{\R}{\ensuremath{\mathbb{R}}}
\newcommand{\C}{\ensuremath{\mathbb{C}}}
\newcommand{\Z}{\ensuremath{\mathbb{Z}}}
\newcommand{\Q}{\mathbb{Q}}
\newcommand{\N}{\mathbb{N}}
\newcommand{\W}{\mathbb{W}}
\newcommand{\Qoft}{\mathbb{Q}(t)}  %use in linux

\newcommand\frakfamily{\usefont{U}{yfrak}{m}{n}}
\DeclareTextFontCommand{\textfrak}{\frakfamily}
\newcommand\G{\textfrak{G}}

% Fractions
\newcommand{\fof}{\frac{1}{4}}  %oneforth
\newcommand{\foh}{\frac{1}{2}}  %onehalf
\newcommand{\fot}{\frac{1}{3}}  %onethird
\newcommand{\fop}{\frac{1}{\pi}}    %1/pi
\newcommand{\ftp}{\frac{2}{\pi}}    %2/pi
\newcommand{\fotp}{\frac{1}{2 \pi}} %1/2pi
\newcommand{\fotpi}{\frac{1}{2 \pi i}}
\newcommand{\cm}{c_{\text{{\rm mean}}}}
\newcommand{\cv}{c_{\text{{\rm variance}}}}

% Theorem / Lemmas et cetera

%\theoremstyle{definition}
\newcommand{\vars}[2]{ #1_1, \dots, #1_{#2} }
\newcommand{\ncr}[2]{{#1 \choose #2}}
\newcommand{\twocase}[5]{#1 \begin{cases} #2 & \text{{\rm #3}}\\ #4
&\text{{\rm #5}} \end{cases}   }
\newcommand{\threecase}[7]{#1 \begin{cases} #2 & \text{{\rm #3}}\\ #4
&\text{{\rm #5}}\\ #6 & \texttt{{\rm #7}} \end{cases}   }
\newcommand{\twocaseother}[3]{#1 \begin{cases} #2 & \text{#3}\\ 0
&\text{otherwise} \end{cases}   }

%Formatting
\renewcommand{\baselinestretch}{1}
\newcommand{\murl}[1]{\href{mailto:#1}{\textcolor{blue}{#1}}}
\newcommand{\hr}[1]{\href{#1}{\url{#1}}}

%%% NEW COMMANDS FOR THIS PAPER
\newcommand{\dfq}{d_{\rm FQ}}
\newcommand{\dave}{d_{\rm FQ; ave}}
\newcommand{\daven}{d_{\rm FQ; ave}(n)}
\newcommand{\todo}[1]{\textcolor{red}{\textbf{#1}}}
\newcommand{\DN}[1]{\textcolor{blue}{\textbf{(DN:#1)}}}
\newcommand{\PF}[1]{\textcolor{cyan}{\textbf{(PF:#1)}}}

%%% NEW COMMANDS FROM VARIANCE
\newcommand{\PP}[1]{\mathbb{P}[#1]}
\newcommand{\E}[1]{\mathbb{E}[#1]}
\newcommand{\V}[1]{\text{{\rm Var}}[#1]}
\newcommand{\BS}{\mathcal{S}}
\newcommand{\T}{\mathcal{T}}
\newcommand{\LT}{\mathcal{L_T}}
\newcommand{\LS}{\mathcal{L_S}}
\newcommand{\ZS}{\mathcal{Z_S}}
\newcommand{\ds}{\displaystyle}
\newcommand{\nsum}{Y_n}
